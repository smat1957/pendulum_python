\chapter{支点が鉛直に振動している単振子}

\section{モデルの定式化}

\begin{comment}
    \begin{figure}[htbp]
        \begin{minipage}[b]{0.45\linewidth}
          \centering
          \includegraphics[keepaspectratio, scale=0.6]{vertical.eps}
          \caption{支点が鉛直に振動している単振子}
        \end{minipage}
        %\begin{minipage}[b]{0.45\linewidth}
          %\centering
          %\includegraphics[keepaspectratio, scale=0.35]{p13-2.eps}
          %\caption{p13-2}
        %\end{minipage}
      \end{figure}
\end{comment}

支点が鉛直方向に$a\cos\gamma t$にしたがって振動している単振子

単振子の座標は、$(l\sin\varphi,l\cos\varphi)$なので、質点$m$の座標は、$x= l\sin\varphi, y = a\cos\gamma t + l\cos\varphi$

次の準備をしておいて

\begin{align*}
   \dot{x} &= l\dot{\varphi}\cos\varphi \quad , \quad \dot{y}=-a\gamma\sin\gamma t - l\dot{\varphi}\sin\varphi\\
   \dot{x}^2 &= l^2\dot{\varphi}^2\cos^2\varphi \quad , \quad \dot{y}^2 = a^2\gamma^2\sin^2\gamma t +2a\gamma l \dot{\varphi}\sin\gamma t \sin\varphi + l^2\dot{\varphi}^2\sin^2\varphi
\end{align*}

運動エネルギー$T$とポテンシャル・エネルギー$U$は次の通り
($\sin^2\alpha+\cos^2\alpha=1$)

\begin{align*}
   T &= \frac{m}{2}\left(\dot{x}^2+\dot{y}^2\right)\\
   &=\frac{m}{2}\left(l^2\dot{\varphi}^2\left(\sin^2\varphi+\cos^2\varphi\right) +a^2\gamma^2\sin^2\gamma t + 2a\gamma l \dot{\varphi}\sin\gamma t\sin\varphi\right)\\
   &=\frac{m}{2}l^2\dot{\varphi}^2 + ma\gamma l\dot{\varphi}\sin\gamma t\sin\varphi + \frac{m}{2}a^2\gamma^2\sin^2\gamma t\\
   U &= -mgy\\&=-mg(a\cos\gamma t +l\cos\varphi)\\&=-mga\cos\gamma t - mgl\cos\varphi
\end{align*}

ラグランジアン$L=T-U$を求める.

\begin{align*}
   L &= T-U\\
   &=\frac{m}{2}l^2\dot{\varphi}^2 + ma\gamma l\dot{\varphi}\sin\gamma t\sin\varphi + \frac{m}{2}a^2\gamma^2\sin^2\gamma t + mga\cos\gamma t + mgl\cos\varphi\\
   &=\frac{m}{2}l^2\dot{\varphi}^2 +\left(ma\gamma^2l\cos\gamma t\cos\varphi -ma\gamma l\frac{d}{dt}(\cos\varphi\sin\gamma t)\right)\\
   &\qquad \qquad \qquad \qquad\quad \quad  +\frac{m}{2}a^2\gamma^2\sin^2\gamma t + mga\cos\gamma t +mgl\cos\varphi\\&=\frac{m}{2}l^2\dot{\varphi}^2+ma\gamma^2l\cos\gamma t\cos\varphi + mgl\cos\varphi\\
   &\qquad \qquad +\left(-ma\gamma l\frac{d}{dt}(\cos\varphi\sin\gamma t) + \frac{m}{2}a^2\gamma^2\sin^2\gamma t + mga\cos\gamma t \right)
\end{align*}

ここで、次の関係を使っている.
\begin{align*}
   -ma\gamma l\frac{d}{dt}(\cos\varphi\sin\gamma t) &= -ma\gamma l\left( -\dot{\varphi}\sin\varphi\sin\gamma t +\gamma\cos\varphi\cos\gamma t \right)\\
   &= ma\gamma l\dot{\varphi}\sin\gamma t\sin\varphi - ma\gamma^2l\cos\gamma t\cos\varphi
\end{align*}

更に、次の関数$f(\varphi,t)$を用意すると、

\[f(\varphi,t)=-ma\gamma l\cos\varphi\sin\gamma t + \displaystyle\frac{1}{4}ma^2\gamma^2t - \frac{1}{8}ma^2\gamma\sin 2\gamma t + mga\frac{1}{\gamma}\sin\gamma t\]

ラグランジアンの最後の括弧内の部分は、関数$f(\varphi,t)$の時間に関する完全導関数になっている.($\cos2\alpha=1-2\sin^2\alpha$)

\begin{align*}
   \frac{df(\varphi,t)}{dt}&=-ma\gamma l(-\dot{\varphi}\sin\varphi\sin\gamma t + \gamma\cos\varphi\cos\gamma t) +\frac{1}{4}ma^2\gamma^2 - \frac{1}{8}ma^2\gamma 2\gamma \cos2\gamma t + mga\frac{1}{\gamma}\gamma\cos\gamma t\\
   &= -ma\gamma l\frac{d}{dt}(\cos\varphi\sin\gamma t) + \frac{1}{4}ma^2\gamma^2 - \frac{1}{4}ma^2\gamma^2(1-2\sin^2\gamma t) + mga\cos\gamma t\\
   &= -ma\gamma l\frac{d}{dt}(\cos\varphi\sin\gamma t) + \frac{1}{4}ma^2\gamma^2 2\sin^2\gamma t + mga\cos\gamma t\\
   &= -ma\gamma l\frac{d}{dt}(\cos\varphi\sin\gamma t) + \frac{m}{2}a^2\gamma^2\sin^2\gamma t +mga\cos\gamma t
\end{align*}

従って、最終的なLagrangianは、(時間に関する完全導関数の部分を除いて)

\[L=\displaystyle\frac{ml^2}{2}\dot{\varphi}^2 + ma\gamma^2l\cos\gamma t\cos\varphi + mgl\cos\varphi\]

次の計算から

\begin{align*}
   \displaystyle\frac{\partial L}{\partial \dot{\varphi}}&=ml^2\dot{\varphi}\\
   \frac{\partial L}{\partial \varphi}&=-mla\gamma^2\cos\gamma t\sin\varphi -mgl\sin\varphi
\end{align*}

Euler-Lagrange eq.は

\begin{align*}
   \ddot{\varphi}=-\frac{a\gamma^2}{l}\cos\gamma t\sin\varphi - \frac{g}{l}\sin\varphi
\end{align*}

\section{Pythonによる模擬実験}

これを連立の一階微分方程式に直すと

\begin{align*}
   \frac{\mathrm{d}\varphi}{\mathrm{d}t} &= \dot{\varphi}\\
   \frac{\mathrm{d}\dot{\varphi}}{\mathrm{d}t} &= -\frac{a\gamma^2}{l}\cos\gamma t\sin\varphi - \frac{g}{l}\sin\varphi
\end{align*}

質点の座標は

\[x=l\sin\varphi \qquad,\quad y=a\cos\gamma t + l\cos\varphi\]

\lstset{escapechar=@,style=custompy}
\lstinputlisting[caption=支点が鉛直に振動している単振子,label=pythonProgram5]{vertical.py}
